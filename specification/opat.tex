\documentclass{article}
\usepackage[a4paper,margin=1in]{geometry}
\usepackage{amsmath, amssymb, listings, hyperref}
\usepackage{booktabs, longtable}
\usepackage[T1]{fontenc}

\title{\textbf{OPAT File Format Specification}\\Version 1.0}
\author{Emily M. Boudreaux}
\date{February 2025}

\begin{document}

\maketitle

\begin{abstract}
The untitled 4DSSE code (hereafter ``the code'') will make use of the OPAT (Open
  Parameterized Array Table) file format for efficient storage and retrieval of
  double precision tabular data parameterized by some arbitrary length vector
  (such as a composition vector). The format is designed to be extensible,
  self-contained, and computationally efficient, supporting fast lookups,
  structured metadata, and data integrity verification. We have developed the
  OPAT format specifically for opacity tables; however, we are releasing it
  generally under GPLv3 and encourage its use  for other applications. This
  document contains a full description of OPAT, including data types,
  endianness, byte arrangement, and interface tools. 
\end{abstract}

\newpage

\section{Introduction}
The OPAT format provides a structured binary format for storing tabular data
sets indexed by an arbitrary floating point vector (for example hydrogen mass
fraction ($X$) and metal mass fraction ($Z$)).  It is designed with
high-performance, data-integrity, and generalizability in mind.  OPAT is the
format that all radiative opacity tables are stored in for the code.  The format
ensures data integrity using SHA-256 checksums.  All numeric values are stored
as 64-bit floating point (double) values;  this ensures approximately 15
decimals of precision.  Data are explicitly stored in \textbf{little endian}
format. 
\noindent Each OPAT file consists of three main sections:
\begin{enumerate}
    \item \textbf{File Header}: Contains metadata, format version, and a lookup table offset. 
    \item \textbf{Data Cards}: Contains data cards. These are collections of tables and other data associated to index vectors. 
    \item \textbf{Card Catalog}: Maps indexing vector values to byte offsets for direct access. 
\end{enumerate}

\noindent A few important things to note here are that the table index is
stored at the \textbf{end} of the file.  Therefore when reading, the index
offset will need to be used in order to move the current position to the end of
the file.  This is done for ease of writing since the checksums of each data
card will not be known until the cards have been read in.  Each data card contains
its own header which outlines its structure.  These are discussed in detail in
\S \ref{Sec:Cards}. 

\section{OPAT File Structure}
An OPAT file is structured as follows:

\begin{center}
\begin{tabular}{|l|l|l|}
\hline
\textbf{Section} & \textbf{Size (bytes)} & \textbf{Description} \\
\hline
File Header & 256 (fixed) & Metadata, format version, index offset \\
Data Cards & Variable & A Header, lookup index, and arbitrary numeric data \\
Card Catalog & Variable & Maps indexing vector to byte locations of data card\\
\hline
\end{tabular}
\end{center}

\subsection{File Header}
The header provides general metadata and file organization details.
\begin{longtable}{|l|l|l|l|}
\hline
\textbf{Field} & \textbf{Type} & \textbf{Size (bytes)} & \textbf{Description} \\
\hline
Magic Number & char[4] & 4 & "OPAT" identifier \\
Version & uint16 & 2 & Format version (e.g., 1.0 = 0x0001) \\
Num Cards & uint32 & 4 & Number of stored data cards \\
Header Size & uint32 & 4 & Byte offset to the Data Cards \\
Catalog Offset & uint64 & 8 & Byte offset of card catalog \\
Creation Date & char[16] & 16 & Creation Date (Feb 15, 2025) \\
Source Info & char[64] & 64 & Description of data source \\
Comment & char[128] & 128 & Units, notes, etc. \\
Num Indices & uint16 & 2 & Number of index values per table \\
Reserved & char[24] & 24 & Future use (zero-filled) \\
\hline
\end{longtable} 

\subsection{Table Index}
Each entry in the catalog provides byte offsets for locating the
corresponding data card.  Note that the index vector size may be
variable from OPAT file to OPAT file, but within one OPAT file it must always be
constant.  Data cards can be indexed with vectors of size anywhere between 1 and
255 floats.  This might look like a table for each X, Z pair (as in OPAL type I
files) or a vector of X, Z, excess O, and excess C (as in OPAL type II files). 
This must be specified before adding any tables.  \textbf{All tables in a single
OPAT file must use an index vector of the same size or reading will fail}.  The python
\texttt{opatio} module enforces this behavior through the
\texttt{OpatIO.set\_numIndex} method. 

\noindent\textbf{N.B.} If duplicate index vectors are given then only the most
recent card associated to an index vector will be stored.  The python module
has safe checking for this (explicitly disallowing duplicate index vectors). 
\begin{longtable}{|l|l|l|l|}
\hline
\textbf{Field} & \textbf{Type} & \textbf{Size (bytes)} & \textbf{Description} \\
\hline
Index $n$ \texttt{n} & float64 & 8 & $n^{th}$ index of file \\
..... & float64 & 8 & $(n+1)^{th}$ index \\
Byte Start & uint64 & 8 & Start of data card in file \\
Byte End & uint64 & 8 & End of data card in file \\
Checksum & char[32] & 32 & SHA-256 checksum of data card \\
\hline
\end{longtable} 

Each entry is 48 + 8*numIndex bytes long, allowing for efficient binary
searching.  The minimum possible size is 56 bytes per index, while the maximum
possible size is 2088 bytes per index. 

\section{Data Cards}\label{Sec:Cards}
Each data card is formatted similarly to the overall OPAT table.  Each card contains a header, an index, and a set of tables.  The structure of a data card is as follows: 
\begin{enumerate}
  \item \textbf{Card Header}: Contains metadata and an offset to the Card Index
  \item \textbf{Tables}: Contains the actual data stored in the card
  \item \textbf{Card Index}: Contains byte offsets, relative to the start of the data card, for the locations of tables.  Tables are indexed by tags which are 8-byte unsigned character arrays. 
\end{enumerate}

\subsection{Data Card Header}
Each data card has a header not unlike the global header for the entire OPAT file.  This has the following form: 
\begin{longtable}{|l|l|l|p{5cm}|}
\hline
\textbf{Field} & \textbf{Type} & \textbf{Size (bytes)} & \textbf{Description} \\
\hline
Magic Number & char[4] & 4 & "CARD" identifier \\
Num Tables & uint32 & 4 & Number of stored tables \\
Header Size & uint32 & 4 & Byte offset to the Data Tables relative to the start of the data card \\
Index Offset & uint64 & 8 & Byte offset of card Index relative to the start of the data card \\
Card Size & uint64 & 8 & Total byte size of the data card \\
Comment & char[128] & 128 & Units, notes, etc. \\
Reserved & char[100] & 100 & Future use (zero-filled) \\
\hline
\end{longtable} 
Note that the header is 256 bytes. 

\subsection{Data Card Index}
The data card index is written immediately after the header (this is in contrast
to the Card Catalog for the entire OPAT file which is the last part of the
file).  It contains a variable number of bytes depending on how many tables are
present in the file.  The data card index is a table with 7 columns and a
number of rows equal to the Num Tables field in the header.  These data are
arranged in row-major order in the file.  Each row has the form: 

\noindent\textbf{N.B.} The same tag can exist in multiple data cards;  however,
within one data card there may only be one instance of the same tag.  Tags are
case insensitive. 
\begin{longtable}{|l|l|l|p{5cm}|}
\hline
\textbf{Field} & \textbf{Type} & \textbf{Size (bytes)} & \textbf{Description} \\
\hline
  Tag & char[8] & 8 & tag used to identify the table in the card.  \\
  Byte Start & uint64 & 8 & Start byte of the table relative to the start byte of the card.  The start byte of the card is byte 0.  \\
  Byte End & uint64 & 8 & End byte of the table relative to the start byte of the card.  The start byte of the card is byte 0.  \\
  Num Columns & uint16 & 2 & Number of columns the table contains.  \\
  Num Rows & uint16 & 2 & Number of rows the table contains.  \\
  Column Name & char[8] & 8 & Label for value parameterizing columns.  \\
  Row Name & char[8] & 8 & Label for value parameterizing rows.  \\
  Reserved & char[20] & 20 & Reserved for future use.  \\
\hline
\end{longtable}

\subsection{Data card tables}
The actual data in an OPAT file is stored in tables.  These are row-major lists
of double precision values.  Each table has a known size from the data card
index and so reading in is simply a matter of reading the correct number of
bytes and rearranging to fit the number of rows x number of columns. 

\section{Checksum and Data Integrity}
Each data card is assigned a SHA-256 checksum stored in the Card Catalog for
validation.  These should be checked whenever reading in OPAT files (the
official OPAT libraries handle this automatically.) 

\section{Creation}
The python module \texttt{opatio} (in \texttt{utils/opatio}) provides
a straightforward interface for the creation and reading of OPAT formatted
files.  There is a README in that directory which outlines its use. 

\section{Usage}
Both the python module \texttt{opatio} and the C++ module \texttt{opatIO} allow for OPAT file generation. 

\end{document}
