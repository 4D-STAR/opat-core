\documentclass{article}
\usepackage[a4paper,margin=1in]{geometry}
\usepackage{amsmath, amssymb, listings, hyperref}
\usepackage{booktabs, longtable}
\usepackage[T1]{fontenc}
\usepackage{inconsolata}

\title{\textbf{OPAT File Format Specification}\\Version 1.0}
\author{Emily M. Boudreaux}
\date{February 2025}

\begin{document}

\maketitle

\begin{abstract}
The untitled 4DSSE code (herafter ``the code'' will make use of the OPAT
  (Opacity Table) file format for efficient storage and retrieval of opacity
  tables parameterized in density and temperature. The format is designed to be
  extensible, self-contained, and computationally efficient, supporting fast
  lookups, structured metadata, and data integrity verification. All radiative
  opacity tables used by the code should be stored in OPAT format. Python
  bindings are provided to allow users to easily generate OPAT formated files.
\end{abstract}

\newpage

\section{Introduction}
The OPAT format provides a structured binary format for storing opacity tables
indexed by an arbitrary floating point vector (for example hydrogen mass
fraction ($X$) and metal mass fraction ($Z$)). It is designed for
high-performance, data-integrity, and generalizability in mind. OPAT is the
format that all radiative opacity tables are stored in for the code. The format
ensures data integrity using SHA-256 checksums. All numeric values are stored
as 64 bit floating point (double) values; this ensures approximately 15
decimals of precision. The file format is open source and full described in
this document. Data are stored in \textbf{little endian} format. 

\noindent Each OPAT file consists of three main sections:
\begin{enumerate}
    \item \textbf{File Header}: Contains metadata, format version, and a lookup table offset.
    \item \textbf{Data Tables}: Contains gridded opacity data.
    \item \textbf{Table Index}: Maps indexing vector values to byte offsets for direct access.
\end{enumerate}

\noindent A few important things to note here are that the table index is
stored at the \textbf{end} of the file. Therefore when reading, the index
offset will need to be used in order to move the current position to the end of
the file. This is done for ease of wrigint since the checksums of each table
wont be known till the tables have been read in. Further, each table is
intended to store opacity values as a function of log density ($\log R$) and
log temperature ($\log T$). The actual numeric values in there could be
anything, however, the more important takeaway is that each table stores its
horizontal and vertical parameters as lists of length n and m and then its data
as a list of length nxm.

\section{File Structure}
An OPAT file is structured as follows:

\begin{center}
\begin{tabular}{|l|l|l|}
\hline
\textbf{Section} & \textbf{Size (bytes)} & \textbf{Description} \\
\hline
File Header & 256 (fixed) & Metadata, format version, index offset \\
Data Tables & Variable & Opacity tables for different compositions \\
Table Index & Variable & Maps indexing vector to byte locations \\
\hline
\end{tabular}
\end{center}

\subsection{File Header}
The header provides general metadata and file organization details.

\begin{longtable}{|l|l|l|l|}
\hline
\textbf{Field} & \textbf{Type} & \textbf{Size (bytes)} & \textbf{Description} \\
\hline
Magic Number & char[4] & 4 & "OPAT" identifier \\
Version & uint16 & 2 & Format version (e.g., 1.0 = 0x0001) \\
Num Tables & uint32 & 4 & Number of stored ($X, Z$) tables \\
Header Size & uint32 & 4 & Byte offset to the Data Tables \\
Index Offset & uint64 & 8 & Byte offset of Table Index \\
Creation Date & char[16] & 16 & Creation Date (Feb 15, 2025) \\
Source Info & char[64] & 64 & Description of data source \\
Comment & char[128] & 128 & Units, notes, etc. \\
Nun Indices & uint16 & 2 & Number of index values per table \\
Reserved & char[24] & 24 & Future use (zero-filled) \\
\hline
\end{longtable}

\subsection{Table Index}
Each entry in the index provides byte offsets for locating the corresponding
opacity table. Note that the table index size is variable OPAT file to OPAT
file but within one OPAT t file must always be constant. This changes if the
OPAT files has a different number of indexes. Tables can be indexed by anywhere
between 1 and 255 floats. This might look like a table for each X, Z pair (as
in OPAL type I files) or an vector of X, Z, excess O, and excess C (as in OPAL
type II files). This must be specified before adding any tables. \textbf{All
tables in a single OPAT file must use the same number of indices or reading
will fail}. The python \texttt{opatio} module enforces this behavior through
the \texttt{OpatIO.set\_numIndex} method.
\begin{longtable}{|l|l|l|l|}
\hline
\textbf{Field} & \textbf{Type} & \textbf{Size (bytes)} & \textbf{Description} \\
\hline
Index $n$ \texttt{n} & float64 & 8 & $n^{th}$ index of file \\
..... & float64 & 8 & $(n+1)^{th}$ index \\
Byte Start & uint64 & 8 & Start of data in file \\
Byte End & uint64 & 8 & End of data in file \\
Checksum & char[32] & 32 & SHA-265 checksum of table \\
\hline
\end{longtable}

Each entry is 48+8*numIndex bytes long, allowing for efficient binary
searching. The minimum possible size is 56 bytes per index, while the maximum
possible size is 2088 bytes per index.

\subsection{Data Tables}
Each opacity table contains gridded values of opacity as a function of density and temperature.

\textbf{Binary Storage Format:}

\begin{lstlisting}[language=C, basicstyle=\ttfamily]
struct OpacityTable {
    uint32_t N_R;    // Number of density points
    uint32_t N_T;    // Number of temperature points
    double* logR; // Log Density grid
    double* logT; // Log Temperature grid
    double* logK; // Log Opacity values
};
\end{lstlisting}


\section{Checksum and Data Integrity}
Each data table is assigned a SHA-256 checksum stored in the Table Index for validation.

\section{Creation}
The python module opatio (in utils/opatio) provides
a straightforward interface for the creation and reading of opat formated
files. There is a readme in that directory which outlines its use.  

\section{Usage}
Both the python module opatio and the cpp module opatIO allow for opat file generation.


\end{document}

